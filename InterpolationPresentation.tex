\documentclass{beamer}

\usetheme{Copenhagen}

\usecolortheme{dolphin}

% PACKAGES ---------------------------------------------------------------

\usepackage{amsfonts,amsmath,graphicx,subfigure}

% ADD YOUR OWN PACKAGES HERE ---------------------------------------------

%\usepackage{someotherpackage}

\usepackage{lmodern,bbm}


\newcommand{\STRR}{\mathbb{R}}

\newcommand{\STRnorm}[1]{\left\Vert#1\right\Vert}

\newcommand{\STRabs}[1]{\left\vert#1\right\vert}

\newcommand{\STRset}[1]{\{#1\}}

\newtheorem{prob}{Problem}

\newtheorem{prop}{Proposition}

\beamertemplatenavigationsymbolsempty % Get rid of the navigation menu

%\newtheorem{lemma}{Lemma}

\newtheorem{cor}{Corollary}

\mode<presentation>

{

\setbeamertemplate{headline}{}

\usecolortheme{default}

}

\title{Curve Fitting and Interpolation}

\author{Stephen Rowe}


\date{\today}


\begin{document}

 

\begin{frame}

\titlepage

\end{frame}
 


\section{Why am I Here Listening to This?}
\begin{frame}
\frametitle{People Like Pictures}
\begin{itemize}
\item Let's say someone gives us some points $\{x_i\}_{i=1}^n$ and some corresponding values $\{y_i\}_{i=1}^N$.
\item They ask us to "draw something that fits through the data points"
\item They hope that they can then choose a new $x$ point and guess (meaningfully) a corresponding $y$ value.
\item They might demand that the curve we draw passes through every single $(x_i,y_i)$ point.
\item They might be a bit more lax and say ``do your best and try to approximate each $(x_i,y_i)$ point."
\end{itemize}
\end{frame}




\end{document}

\begin{frame}

\frametitle{}

\begin{itemize}

\end{itemize}

\end{frame}